\documentclass[a4paper, 12pt, twocolumn]{article}

% Paquetes principales
\usepackage[utf8]{inputenc}     % Codificación UTF-8
\usepackage{graphicx}           % Imágenes
\usepackage{amsmath, amssymb}   % Matemáticas avanzadas
\usepackage{geometry}           % Control de márgenes
\usepackage{hyperref}           % Hipervínculos
\usepackage[sort, numbers]{natbib}    % Citas y bibliografía
\usepackage{subfiles}           % Para dividir el documento en subarchivos
\usepackage{lipsum}             % Para generar lorem ipsum
\usepackage[spanish]{babel}
\usepackage{listings}
\usepackage{pdfpages}
\usepackage{silence}
\usepackage{csquotes}
\usepackage[outputdir=aux]{minted}
\usepackage{dracula-theme}

\geometry{margin=0.5in}           % Márgenes de 1 pulgada
\graphicspath{ {./images/} }

% Título y autor
\title{Generación de imágenes con redes neuronales generativas adversarias profundas}
\author{Rodrigo Dávalos Alarcón}
\date{\today} % Fecha automática

% Documento
\begin{document}

\sloppy

\maketitle

\begin{abstract}
    Este es el resumen de la investigación. Breve descripción de los objetivos, metodología y resultados.
\end{abstract}

\subfile{chapters/introduccion.tex}
\subfile{chapters/marco_teorico.tex}
\subfile{chapters/metodologia.tex}
\subfile{chapters/implementacion.tex}
\subfile{chapters/resultados.tex}
\subfile{chapters/discusion.tex}
\subfile{chapters/conclusiones.tex}

% Bibliografía

\bibliographystyle{unsrtnat}
\bibliography{literature/references}

\end{document}

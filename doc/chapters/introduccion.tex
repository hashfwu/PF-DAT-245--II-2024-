\documentclass[../main.tex]{subfiles}
\begin{document}

\section{Introducción}

La clasificación de imágenes es una parte importante de la inteligencia artificial. Y si se desarrolla lo suficiente, la clasificación de imágenes puede llegar a reconcer objetos en tiempo real, reconcer patrones, ser inclusive mejor que los humanos en ciertas tareas. Sus aplicaciones pueden ir desde la medicina hasta la conducción de vehiculos autónomos. En esta investigación veremos la clasificación de pinturas y se estudiará como es que un conjunto de redes convolucionales puede lograr reconcer patrones que a los humanos nos puede ser dificiles de explicar (y por ende de programar).

\subsection{Contexto}

En la actualidad las inteligencias artificiales han avanzado demasiado. Hoy lo que mas impacto genera sin duda son las inteligencias artificilaes generativas. Las inteligencias artificiales para clasificación y predicción existen desde hace mucho tiempo. Uno de los primeros algoritmos para clasificación y predicción es  el algoritmo de Native Bayes \cite{webb2010naive}. Este algoritmo puede funcionar para clasificación binaria y de multiples clases. Este algoritmo data del año 1760. Uno de los primeros programas de clasificación que usan redes convolucionales data del año 1980 llamado LeNet usado para el reconocimiento de dígitos \cite{kayed2020classification}. De hecho existen muchas arquitecturas para las CNN's\cite{swapna2020cnn}. Entre ellas se encuentran:

\begin{itemize}
    \item Alex Net
    \item Le Net
    \item VGG
    \item Google Net
    \item Res Net
\end{itemize}

Es elección de cada persona escojer el modelo que más le convenga o adaptar uno ya creado (crear un modelo suele ser más difícil).

El modelo que usaremos en esta investigación se basa en el modelo usado por el usuario \textbf{mkkoehler} de la plataforma \url{www.kaggle.com} \cite{Mkkoehler_2020}. Este modelo es un modelo simple de redes convolucionales que ayudan a ejemplificar como funcionan. 

\subsection{Motivación}

La principal motivación de esta investigación radica en la curiosidad por explorar la relación entre las inteligencias artificiales y las imágenes. La generación de imágenes es, sin duda, uno de los temas más fascinantes en este campo de investigación. Sin embargo, antes de adentrarse plenamente en el proceso de generación, resulta fundamental comprender a fondo el funcionamiento de las redes neuronales convolucionales. Este es, precisamente, otro de los objetivos de este proyecto: conocer cómo operan las redes convolucionales y cómo una máquina es capaz de identificar patrones. A partir de este entendimiento, se busca, en un futuro cercano, comprender cómo estas redes pueden generar nuevas imágenes basándose en los patrones previamente aprendidos.

\subsection{Objetivo general}

Comprender el funcionamiento de una red neuronal convolucional y su aplicación en la clasificación de imágenes.

\subsection{Objetivos específicos}

\begin{enumerate}
    \item Analizar el funcionamiento de una red neuronal convolucional.
    \item Implementar el código de una red neuronal convolucional para la clasificación de imágenes.
    \item Interpretar los resultados obtenidos con el fin de extraer conclusiones significativas.
    \item Explicar el proceso de clasificación de imágenes utilizando una red neuronal convolucional. \item Proponer mejoras al código previamente implementado.
    \item Modificar el código existente para optimizar su rendimiento.
    \item Incorporar los métodos sugeridos por el docente, como el Análisis de Componentes Principales (PCA) y el aprendizaje no supervisado, y justificar su aplicación o la falta de aplicabilidad.
\end{enumerate}

\subsection{Estructura del documento}

Este documento contendrá no solo texto, si no también ejecución de código para que se tenga un documento sólido y que elimine la necesidad de ir de los archivos \texttt{.pdf} al \texttt{.ipynb}. Sin embargo se proporcionará los links respectivos al repositorio cuando sea necesario.

\end{document}
